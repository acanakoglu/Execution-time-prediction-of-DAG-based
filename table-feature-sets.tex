

\begin{table}
\caption{Feature-set categorization}


\centering
\label{tab:feature-sets}

\resizebox{1\columnwidth}{!}{
\begin{tabular}{|c|c|c|}
    \hline
 & Basic & Full \\
    \hline
    \multirow{3}{*}{Black-box} & Input Data (BBI) &  BBI $\cup$ BBT $\cup$ BBT \\ 
    &Task Parameters (BBT) &  Composite Features (BBC) \\
     &Execution Environment (BBE)  &   \\
     \hline
    \multirow{3}{*}{Gray-box} & Input Data (GBI) & GBI $\cup$ GBT $\cup$ GBT \\ 
    &Task Parameters (GBT) &  Composite Features (GBC) \\
     &Execution Environment (GBE)  &   \\
    \hline
\end{tabular}
}

\vspace{0.2cm}

\centering
\caption{Example of features in the four categories}
\label{tab:feature-sets-instance}

\resizebox{1\columnwidth}{!}{
\begin{tabular}{|c|c|c|}
    \hline
 & Basic & Full \\
    \hline
    \multirow{3}{*}{Black-box} & BBI = input-size, num-files &  BBI $\cup$ BBT $\cup$ BBT \\ 
    &BBT = arg1, arg2 $\cdots$ &  BBC = input-size/cores, log(cores) \\
     &BBE = memory, cores  &   \\
     \hline
    \multirow{3}{*}{Gray-box} & GBI = BBI + interval-length & GBI $\cup$ GBT $\cup$ GBT \\ 
    &GBT  min-distance & GBC = BBC + input-size/num-tasks \\
     &GBE = BBE + num-tasks  &   \\
    \hline
\end{tabular}
}


\end{table}